%%%%%%%%%%%%%%%%%%%%%%%%%%%%%%%%%%%%%%%%%%%%%%%%%%%%%%%%%%%%%%%%%%%%%%%%%%%%%%%%
% Papírméret, betűméret, margó, magyar karakterek                              %
%%%%%%%%%%%%%%%%%%%%%%%%%%%%%%%%%%%%%%%%%%%%%%%%%%%%%%%%%%%%%%%%%%%%%%%%%%%%%%%%

\documentclass[a4paper,12pt]{article}
\special{papersize=210mm,297mm}

\usepackage{anysize}
\marginsize{2.5cm}{2.5cm}{2.5cm}{2.5cm}

\usepackage[utf8]{inputenc}
\usepackage[magyar]{babel}

%%%%%%%%%%%%%%%%%%%%%%%%%%%%%%%%%%%%%%%%%%%%%%%%%%%%%%%%%%%%%%%%%%%%%%%%%%%%%%%%
% Fedlap inicializálása                                                        %
%%%%%%%%%%%%%%%%%%%%%%%%%%%%%%%%%%%%%%%%%%%%%%%%%%%%%%%%%%%%%%%%%%%%%%%%%%%%%%%%

\usepackage{fedlap}

\csapat{unexpected\_exceptions}{59}
\konzulens{Ferencz Endre}

\taga{Biró Loránd}{NCZAGL}{lol.kylerrr@gmail.com}
\tagb{Kanyó Tibor}{NXWUKE}{kanyo.tibi@gmail.com}
\tagc{Magyar Dániel}{SUFFGT}{samuraidanm@gmail.com}
\tagd{Tarjáni Tamás}{S499KV}{tarjanitomi@gmail.com}
\tage{Vajsz Kornél}{VUYNAW}{roncsipar@gmail.com}

%%%%%%%%%%%%%%%%%%%%%%%%%%%%%%%%%%%%%%%%%%%%%%%%%%%%%%%%%%%%%%%%%%%%%%%%%%%%%%%%
% Fejléc és lábléc                                                             %
%%%%%%%%%%%%%%%%%%%%%%%%%%%%%%%%%%%%%%%%%%%%%%%%%%%%%%%%%%%%%%%%%%%%%%%%%%%%%%%%

\usepackage{fancyhdr}

\setlength{\headheight}{1.4em}
\setlength{\headsep}{2em}

\fancyhf{}
\fancyhead[OL] { \leftmark{} }
\fancyhead[OR] { \tmpcsapat }
\fancyfoot[OC] { \thepage }
\fancyfoot[OR] { \tmpdatum }

\pagestyle{fancy}

%%%%%%%%%%%%%%%%%%%%%%%%%%%%%%%%%%%%%%%%%%%%%%%%%%%%%%%%%%%%%%%%%%%%%%%%%%%%%%%%
% Napló                                                                        %
%%%%%%%%%%%%%%%%%%%%%%%%%%%%%%%%%%%%%%%%%%%%%%%%%%%%%%%%%%%%%%%%%%%%%%%%%%%%%%%%

\usepackage{longtable}

\newenvironment{journal}
{
	\hbadness 10000
	\begin{longtable}{|p{60pt}|l|l|p{216pt}|}
	\hline
	\textbf{Kezdet} & \textbf{Időtartam} & \textbf{Résztvevők} & \textbf{Leírás} \\
	\hline
	\endfirsthead
	\hline
	\textbf{Kezdet} & \textbf{Időtartam} & \textbf{Résztvevők} & \textbf{Leírás} \\
	\hline
	\endhead
}
{
	\end{longtable}
}

\newcommand{\journalentry}[4]
{
	{#1} & {#2} óra & \parbox{50pt}{#3} & {#4} \\
	\hline
}

%%%%%%%%%%%%%%%%%%%%%%%%%%%%%%%%%%%%%%%%%%%%%%%%%%%%%%%%%%%%%%%%%%%%%%%%%%%%%%%%
% UseCase leírás                                                               %
%%%%%%%%%%%%%%%%%%%%%%%%%%%%%%%%%%%%%%%%%%%%%%%%%%%%%%%%%%%%%%%%%%%%%%%%%%%%%%%%

\newenvironment{usecase}
{
	\hbadness 10000
	\begin{longtable}[l]{|p{100pt}|p{328pt}|}
	\hline
	\endfirsthead
	\hline
	\endhead
}
{
	\hline
	\end{longtable}
}

\newcommand{\usecaseentry}[2]
{
	\hline
	\textbf{#1} & {#2}\\
}

%%%%%%%%%%%%%%%%%%%%%%%%%%%%%%%%%%%%%%%%%%%%%%%%%%%%%%%%%%%%%%%%%%%%%%%%%%%%%%%%
% FileList leírás                                                               %
%%%%%%%%%%%%%%%%%%%%%%%%%%%%%%%%%%%%%%%%%%%%%%%%%%%%%%%%%%%%%%%%%%%%%%%%%%%%%%%%

\newenvironment{filelist}
{
	\hbadness 10000

	\begin{longtable}{|p{145pt}|p{35pt}|p{63pt}|p{163pt}|}
	\hline
	\textbf{Fájl neve} & \textbf{Méret} & \textbf{Keletkezés ideje} & \textbf{Tartalom} \\
	\hline
	\endfirsthead
	\hline
	\textbf{Fájl neve} & \textbf{Méret} & \textbf{Keletkezés ideje} & \textbf{Tartalom} \\
	\hline
	\endhead
}
{
	\end{longtable}
}

\newcommand{\filelistentry}[4]
{
	{#1} & {#2} b & {#3} & {#4} \\
	\hline
}
%%%%%%%%%%%%%%%%%%%%%%%%%%%%%%%%%%%%%%%%%%%%%%%%%%%%%%%%%%%%%%%%%%%%%%%%%%%%%%%%
% Egyebek                                                                      %
%%%%%%%%%%%%%%%%%%%%%%%%%%%%%%%%%%%%%%%%%%%%%%%%%%%%%%%%%%%%%%%%%%%%%%%%%%%%%%%%

\usepackage{graphicx}		% Kepek beillesztesehez
\usepackage{epstopdf}		% EPS fajlok felismeresehez
\graphicspath{{Images/}}	% Az Images mappaban keresse a kepeket

\anyag{Összefoglalás}
\datum{2012. május 6.}
\setcounter{section}{13}

%%%%%%%%%%%%%%%%%%%%%%%%%%%%%%%%%%%%%%%%%%%%%%%%%%%%%%%%%%%%%%%%%%%%%%%%%%%%%%%%
% Dokumentum                                                                   %
%%%%%%%%%%%%%%%%%%%%%%%%%%%%%%%%%%%%%%%%%%%%%%%%%%%%%%%%%%%%%%%%%%%%%%%%%%%%%%%%

\begin{document}

\fedlap

\section{Összefoglalás}

\subsection{Projekt összegzése}

A feladatot érdekesnek találtuk, a korábbi évek projektjeit elnézve talán a szerencsésebbek között érezhetjük magunkat. Úgy érezzük, hogy a félév során sikerült sokat fejlődnünk szakmai téren. Új kihívást jelentett egy nagyobb projekt közös munkával való elkészítése, szinte egyikünk sem dolgozott még ezelőtt ily módon. Megtanultuk, hogyan kell a nagyobb feladatokat hatékonyan önálló részekre tördelni, kamatoztathattuk a Szoftvertechnológia tárgyban szerzett tervezési és dokumentálási ismereteinket és belsőséges kapcsolatba kerültünk a csapatmunkát támogató SVN különféle alkalmazásaival. Mindannyiunk számára új ismeretet jelentett a félév során a dokumentációk készítéséhez használt LaTex nyelv, ami későbbi munkáink (pl. Önálló laboratóriumok, Szakdolgozat) írásakor is minden bizonnyal a hasznunkra fog válni.

A félév során különösebb nehézséggel nem szembesültünk, nagyon szerencsésen sikerült a csapatot összeszedni, így olajozottan ment minden. Igyekeztünk a feladatokat egyenlően elosztani, és mindenki kivette a részét a munkából, nem igazán jelentkeztek konfliktusok. Minden csapattag hozzá közelebb álló feladatokat kapott, voltak, akik inkább a dokumentáció készítésében, mások a kódolásban érezték jobban otthon magukat. Ha valamit ki kellene emelnünk legnehezebbként, akkor az talán az egymástól nagy mértékben különböző időbeosztással rendelkező tagok összehangolása lenne. Nehéz volt olyan időpontot találni, amikor mindenki ráér egy-egy közös egyeztetésre, szerencsére a munka megfelelő felosztásának köszönhetően erre csak ritkán volt szükség.

A rendelkezésre álló idő és a feladatokért járó pontszám szerintünk megfelelő, mindig sikerült határidőre elkészülnünk, a nehezebb feladatokra pedig több pontot lehetett kapni, körülbelül arányosan a befektetett munkaórák számával. Különösebb változtatási javaslatunk nincs. Támogatjuk, hogy a fejlesztendő program továbbra is játék legyen, ugyanis egyrészt a kar hallgatóinak nagy része szívesen tölti a szabadidejét akár a jelen feladathoz hasonló játékokkal, másrészt jóval látványosabb és érdekesebb egy játék elkészítése, mint valamilyen komoly szoftveré, ezáltal motiváltabbak a csapatok.


\subsection{Statisztikák}

\subsubsection{A projektre fordított összes munkaidő}

\begin{tabular}{|p{130pt}|c|p{150pt}|}
\hline 
\textbf{Tag neve} & \textbf{Munkaórák száma}\\ 
\hline 
Biró Loránd & 51.5\\ 
\hline 
Kanyó Tibor & 50.2\\
\hline 
Magyar Dániel & 44.5\\
\hline 
Tarjáni Tamás & 50\\
\hline 
Vajsz Kornél & 49.5\\
\hline 
Összesen & 245.7\\
\hline
\end{tabular}

\subsubsection{Forrássorok száma fázisonként}

\begin{tabular}{|p{130pt}|c|p{150pt}|}
\hline 
\textbf{Fázis} & \textbf{Forrássorok száma}\\ 
\hline 
Szkeleton & \\ 
\hline 
Prototípus & \\
\hline 
Grafikus változat & \\
\hline
\end{tabular}

\end{document}