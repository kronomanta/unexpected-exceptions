%%%%%%%%%%%%%%%%%%%%%%%%%%%%%%%%%%%%%%%%%%%%%%%%%%%%%%%%%%%%%%%%%%%%%%%%%%%%%%%%
% Inicializálás                                                                %
%%%%%%%%%%%%%%%%%%%%%%%%%%%%%%%%%%%%%%%%%%%%%%%%%%%%%%%%%%%%%%%%%%%%%%%%%%%%%%%%

%%%%%%%%%%%%%%%%%%%%%%%%%%%%%%%%%%%%%%%%%%%%%%%%%%%%%%%%%%%%%%%%%%%%%%%%%%%%%%%%
% Papírméret, betűméret, margó, magyar karakterek                              %
%%%%%%%%%%%%%%%%%%%%%%%%%%%%%%%%%%%%%%%%%%%%%%%%%%%%%%%%%%%%%%%%%%%%%%%%%%%%%%%%

\documentclass[a4paper,12pt]{article}
\special{papersize=210mm,297mm}

\usepackage{anysize}
\marginsize{2.5cm}{2.5cm}{2.5cm}{2.5cm}

\usepackage[utf8]{inputenc}
\usepackage[magyar]{babel}

%%%%%%%%%%%%%%%%%%%%%%%%%%%%%%%%%%%%%%%%%%%%%%%%%%%%%%%%%%%%%%%%%%%%%%%%%%%%%%%%
% Fedlap inicializálása                                                        %
%%%%%%%%%%%%%%%%%%%%%%%%%%%%%%%%%%%%%%%%%%%%%%%%%%%%%%%%%%%%%%%%%%%%%%%%%%%%%%%%

\usepackage{fedlap}

\csapat{unexpected\_exceptions}{59}
\konzulens{Ferencz Endre}

\taga{Biró Loránd}{NCZAGL}{lol.kylerrr@gmail.com}
\tagb{Kanyó Tibor}{NXWUKE}{kanyo.tibi@gmail.com}
\tagc{Magyar Dániel}{SUFFGT}{samuraidanm@gmail.com}
\tagd{Tarjáni Tamás}{S499KV}{tarjanitomi@gmail.com}
\tage{Vajsz Kornél}{VUYNAW}{roncsipar@gmail.com}

%%%%%%%%%%%%%%%%%%%%%%%%%%%%%%%%%%%%%%%%%%%%%%%%%%%%%%%%%%%%%%%%%%%%%%%%%%%%%%%%
% Fejléc és lábléc                                                             %
%%%%%%%%%%%%%%%%%%%%%%%%%%%%%%%%%%%%%%%%%%%%%%%%%%%%%%%%%%%%%%%%%%%%%%%%%%%%%%%%

\usepackage{fancyhdr}

\setlength{\headheight}{1.4em}
\setlength{\headsep}{2em}

\fancyhf{}
\fancyhead[OL] { \leftmark{} }
\fancyhead[OR] { \tmpcsapat }
\fancyfoot[OC] { \thepage }
\fancyfoot[OR] { \tmpdatum }

\pagestyle{fancy}

%%%%%%%%%%%%%%%%%%%%%%%%%%%%%%%%%%%%%%%%%%%%%%%%%%%%%%%%%%%%%%%%%%%%%%%%%%%%%%%%
% Napló                                                                        %
%%%%%%%%%%%%%%%%%%%%%%%%%%%%%%%%%%%%%%%%%%%%%%%%%%%%%%%%%%%%%%%%%%%%%%%%%%%%%%%%

\usepackage{longtable}

\newenvironment{journal}
{
	\hbadness 10000
	\begin{longtable}{|p{60pt}|l|l|p{216pt}|}
	\hline
	\textbf{Kezdet} & \textbf{Időtartam} & \textbf{Résztvevők} & \textbf{Leírás} \\
	\hline
	\endfirsthead
	\hline
	\textbf{Kezdet} & \textbf{Időtartam} & \textbf{Résztvevők} & \textbf{Leírás} \\
	\hline
	\endhead
}
{
	\end{longtable}
}

\newcommand{\journalentry}[4]
{
	{#1} & {#2} óra & \parbox{50pt}{#3} & {#4} \\
	\hline
}

%%%%%%%%%%%%%%%%%%%%%%%%%%%%%%%%%%%%%%%%%%%%%%%%%%%%%%%%%%%%%%%%%%%%%%%%%%%%%%%%
% UseCase leírás                                                               %
%%%%%%%%%%%%%%%%%%%%%%%%%%%%%%%%%%%%%%%%%%%%%%%%%%%%%%%%%%%%%%%%%%%%%%%%%%%%%%%%

\newenvironment{usecase}
{
	\hbadness 10000
	\begin{longtable}[l]{|p{100pt}|p{328pt}|}
	\hline
	\endfirsthead
	\hline
	\endhead
}
{
	\hline
	\end{longtable}
}

\newcommand{\usecaseentry}[2]
{
	\hline
	\textbf{#1} & {#2}\\
}

%%%%%%%%%%%%%%%%%%%%%%%%%%%%%%%%%%%%%%%%%%%%%%%%%%%%%%%%%%%%%%%%%%%%%%%%%%%%%%%%
% FileList leírás                                                               %
%%%%%%%%%%%%%%%%%%%%%%%%%%%%%%%%%%%%%%%%%%%%%%%%%%%%%%%%%%%%%%%%%%%%%%%%%%%%%%%%

\newenvironment{filelist}
{
	\hbadness 10000

	\begin{longtable}{|p{145pt}|p{35pt}|p{63pt}|p{163pt}|}
	\hline
	\textbf{Fájl neve} & \textbf{Méret} & \textbf{Keletkezés ideje} & \textbf{Tartalom} \\
	\hline
	\endfirsthead
	\hline
	\textbf{Fájl neve} & \textbf{Méret} & \textbf{Keletkezés ideje} & \textbf{Tartalom} \\
	\hline
	\endhead
}
{
	\end{longtable}
}

\newcommand{\filelistentry}[4]
{
	{#1} & {#2} b & {#3} & {#4} \\
	\hline
}
%%%%%%%%%%%%%%%%%%%%%%%%%%%%%%%%%%%%%%%%%%%%%%%%%%%%%%%%%%%%%%%%%%%%%%%%%%%%%%%%
% Egyebek                                                                      %
%%%%%%%%%%%%%%%%%%%%%%%%%%%%%%%%%%%%%%%%%%%%%%%%%%%%%%%%%%%%%%%%%%%%%%%%%%%%%%%%

\usepackage{graphicx}		% Kepek beillesztesehez
\usepackage{epstopdf}		% EPS fajlok felismeresehez
\graphicspath{{Images/}}	% Az Images mappaban keresse a kepeket

\anyag{10. Prototípus beadás}
\datum{2012. április 5.}
\setcounter{section}{9}

%%%%%%%%%%%%%%%%%%%%%%%%%%%%%%%%%%%%%%%%%%%%%%%%%%%%%%%%%%%%%%%%%%%%%%%%%%%%%%%%
% Dokumentum                                                                   %
%%%%%%%%%%%%%%%%%%%%%%%%%%%%%%%%%%%%%%%%%%%%%%%%%%%%%%%%%%%%%%%%%%%%%%%%%%%%%%%%

\begin{document}

\fedlap

\section{Prototípus beadás}

\subsection{Fordítási és futtatási útmutató}

\subsubsection{Fájllista}
Az osztályok package-ekbe, így a fájlok mappákba vannak szervezve. 

Az \textbf{levels} mappa tartalma:
\begin{filelist}
	\filelistentry{01.xml}{726}{2012.04.02. 22:39}{Tesztpálya a játékos mozgásának és a blokkokkal való ütközés-detektálásának helyes működéséhez}
	\filelistentry{02.xml}{755}{2012.04.02. 22:39}{Tesztpálya a ferde blokkokkal való ütközés detektálásához}
	\filelistentry{03.xml}{556}{2012.04.02. 22:39}{Tesztpálya a kulcsok felvételéhez, illetve az ajtó helyes működéséhez}
	\filelistentry{04.xml}{881}{2012.04.02. 22:39}{Tesztpálya a tili-toli mód, továbbá a pályaelemek közötti átjárás ellenőrzéséhez}
	\filelistentry{05.xml}{972}{2012.04.02. 22:39}{Tesztpálya a főpróbához, melyen a játék helyes működésének ellenőrzése a cél}
\end{filelist}

Az \textbf{tests} mappa tartalma:
\begin{filelist}
	\filelistentry{01.txt}{195}{2012.04.02. 22:39}{Forgatókönyv a \emph{Játékos irányítása \#1} tesztesethez}
	\filelistentry{02.txt}{191}{2012.04.02. 22:39}{Forgatókönyv a \emph{Játékos irányítása \#2} tesztesethez}
	\filelistentry{03.txt}{164}{2012.04.02. 22:39}{Forgatókönyv a \emph{Kulcsok és ajtó} tesztesethez}
	\filelistentry{04.txt}{162}{2012.04.02. 22:39}{Forgatókönyv a \emph{Pályarész határok} tesztesethez}
	\filelistentry{05.txt}{124}{2012.04.02. 22:39}{Forgatókönyv a \emph{Főpróba} tesztesethez}
\end{filelist}

\newpage

Az \textbf{src/main} mappa tartalma:
\begin{filelist}
	\filelistentry{ConsoleHelper.java}{588}{2012.03.24. 18:38}{A Windowsos rendszereknél használt CRLF sorvégjelzővel előforduló hibákat kezelő osztály}
	\filelistentry{LevelRenderer.java}{3593}{2012.03.24. 18:38}{A pálya kirajzolásáért felelős osztály}
	\filelistentry{Main.java}{8289}{2012.03.24. 18:38}{A program futását kezelő osztály}
	\filelistentry{StoryboardCommand.java}{416}{2012.03.24. 18:38}{Egy forgatókönyv parancsot reprezentáló osztály}
	\filelistentry{StoryboardReader.java}{2864}{2012.03.24. 18:38}{A beadott teljes forgatókönyvet feldolgozza és utasításokká alakítja }
\end{filelist}

A \textbf{src/model} mappa tartalma:
\begin{filelist}
	\filelistentry{BlockDescriptor.java}{1177}{2012.03.24. 18:38}{A faldarab leíró osztálya}
	\filelistentry{BlockType.java}{81}{2012.03.24. 18:38}{A faldarab típusait tartalmazó osztály}
	\filelistentry{LevelDescriptor.java}{2450}{2012.03.24. 18:38}{Egy pálya leíró osztálya}
	\filelistentry{LevelObjectDescriptor.java}{1050}{2012.03.24. 18:38}{Egy pályán elhelyezhető objektum leírója}
	\filelistentry{LevelObjectType.java}{76}{2012.03.24. 18:38}{A pályán elhelyezhető objektumok típusait tartalmazó osztály}
	\filelistentry{LevelPartDescriptor.java}{1005}{2012.03.24. 18:38}{Egy pályaelem paramétereit tartalmazó osztály}
\end{filelist}

A \textbf{src/gameLogic} mappa tartalma:
\begin{filelist}
	\filelistentry{Block.java}{786}{2012.03.24. 18:38}{A pálya építőelemét megvalósító osztály}
	\filelistentry{Direction.java}{81}{2012.03.24. 18:38}{Az irányokat tartalmazó osztály}
	\filelistentry{DirectionHelper.java}{437}{2012.03.24. 18:38}{A megadott iránynak az ellentettjét visszaadó osztály}
	\filelistentry{Door.java}{448}{2012.03.24. 18:38}{Az ajtót megvalósító osztály}
	\filelistentry{GameObject.java}{862}{2012.03.24. 18:38}{A játékban szereplő objektumok ősosztálya, és a legtöbb tulajdonságukat kezelő osztály}
	\filelistentry{IBounds.java}{1563}{2012.03.24. 18:38}{A blokkok ősosztálya}
	\filelistentry{KeyHolder.java}{644}{2012.03.24. 18:38}{A kulcsot megvalósító osztály}
	\filelistentry{Level.java}{5625}{2012.03.24. 23:55}{A pálya irányítását végző osztály}
	\filelistentry{LevelPart.java}{3966}{2012.03.24. 18:38}{Az egyes pályaelemeket vezérlő osztály}
	\filelistentry{LevelState.java}{301}{2012.03.24. 18:38}{A pálya állapotait tartalmazó felsorolás}
	\filelistentry{Player.java}{4906}{2012.04.02. 22:39}{A játékost kezelő osztály}
	\filelistentry{PlayerJumpState.java}{310}{2012.03.24. 18:38}{A játékos ugrásának állapotait tartalmazó osztály}
	\filelistentry{RectangleBounds.java}{4764}{2012.03.24. 18:38}{A négyszög alakú építőblokkok osztálya}
	\filelistentry{TriangleBounds.java}{3853}{2012.04.02. 22:39}{A háromszög alakú blokkok osztálya}
	\filelistentry{TriangleType.java}{82}{2012.03.24. 18:38}{A ferde talaj dőlésének irányát lehet megadni: balra vagy jobbra lejtsen}
	\filelistentry{Vector2.java}{704}{2012.03.24. 18:38}{2D vektorosztály}
\end{filelist}

A \textbf{src} mappa tartalma:
\begin{filelist}
	\filelistentry{manifest.txt}{61}{2012.03.24. 19:30}{A fordításhoz szükséges leíró fájl}
\end{filelist}

A gyökérben található fájlok:
\begin{filelist}
	\filelistentry{build.bat}{315}{2012.03.24. 19:30}{A fordítást indító fájl}
	\filelistentry{run.bat}{66}{2012.03.24. 19:30}{Az alkalmazást elindító fájl}
\end{filelist}

\newpage

\subsubsection{Fordítás}
A fordítóprogram és annak működési módja megegyezik a 6.1.2 Fordítás nevű fejezetben leírtakkal. A \emph{build.bat} fájlt futtatva a konzolból, előáll a a lefordított jar fájl. A fájl az alábbi utasításokat tartalmazza, melyben csak a jar neve változott:

\begin{verbatim}
@echo off
if not exist bin (mkdir bin)

echo Compiling classes...
javac src/main/*.java src/model/*.java src/gameLogic/*.java -d bin
        -encoding UTF-8

cd bin
echo Creating jar...
jar cfm ../ContinuityPrototype.jar ../src/manifest.txt main/*.class
        model/*.class gameLogic/*.class
cd ..

echo Done
@echo on
\end{verbatim}

\subsubsection{Futtatás}
A futtatás megkönnyítése érdekében létrehoztunk egy batch fájlt: \emph{run.bat}. Ezen fájlt kell futtatni a főkönyvtárból. A fájl maximum négy darab értéket vár indítási paraméternek, ezek választható értékei a \emph{Bemeneti nyelv} fejezet \emph{Indítási argumentumok} nevű részben vannak kifejtve. Ezen fájl az alábbi parancsot végrehajtva elindítja az elkészült jar fájlt: 

\begin{verbatim}
@echo off
java -jar ContinuityPrototype.jar %1 %2 %3 %4
@echo on
\end{verbatim}

Egy példa az indítási paraméterek átadására lehet az alábbi parancssorbeli utasítás, melynek hatására a megadott \emph{test01.txt} forgatókönyvet kezdi el futtatni, és az eredményt pedig a \emph{result.txt} fájlba menti:

\begin{verbatim}
run.bat -a -s -r 2 < tests/test01.txt > result.txt
\end{verbatim}

Szemmel láthatóan több paraméter lett átadva, mint négy, viszont valójában az utolsó két parancs a standard bemenet és kimenet átirányítására vonatkozik, nem pedig az alkalmazásnak szól.


\subsection{A tesztek részletes tervei, leírásuk a teszt nyelvén (kiegészítés a 8. sz. fejezethez)}

Mivel a prototípus lehetőséget ad arra, hogy a beadott forgatókönyvet animáltan rajzoltassuk ki, az elvárt eredmények pontokba inkább az elvárt történések szöveges leírását tettük. Amennyiben megkérdőjelezhető lenne a futás valamely mozzanatának helyessége, lehetőség van a prototípussal a konkrét pozíciók kiírására.

\subsubsection{Játékos irányítása \#1}
\begin{description}
	
	\item[Leírás] A két játékos egy 1x1 pályarészből álló pályán van amikben több össze-vissza elhelyezett téglalap alakú blokk található. A két játékos minden irányban ugrál és mozog.
	
	\item[Ellenőrzött funkcionalitás] Ezzel a teszttel megfigyelhető hogy a játékosok egyszerű mozgatása (jobbra, balra, ugrálás) és az ütközésvizsgálat a téglalap alakú blokkokkal a várt módon történnek.
	
	\item[Várható hibahelyek] Ennél a tesztnél elsősorban a játékosok és a téglalap alakú blokkok ütközése a kritikus pont. A játékosok akármelyik irányból is ütköznek a blokkokkal, semmi képp sem haladhatnak át rajta vagy "lóghatnak" bele, továbbá az hogy már a falhoz érünk miközben a fal felé haladunk nem befolyásolhatja a játékosok függőleges mozgását.

	\item[Pálya]
	\begin{verbatim}
	
		------------
		|K        D|
		|    XX    |
		|   XXXX   |
		|  XXXXXX  |
		|          |
		|XX      XX|
		|          |
		|   XXXX   |
		| 1      2 |
		|XXXXXXXXXX|
		------------	
	\end{verbatim}

	\item[Pálya leíró]
	\begin{verbatim}
	
		<?xml version="1.0" encoding="utf-8"?>
		<Level Width="1" Height="1">
		    <Parts>
		        <LevelPart X="0" Y="0">
		            <Block X="0" Y="9" Width="10" Height="1" />
		            <Block X="3" Y="7" Width="4" Height="1" />
		            <Block X="0" Y="5" Width="2" Height="1" />
		            <Block X="8" Y="5" Width="2" Height="1" />
		            <Block X="2" Y="3" Width="6" Height="1" />
		            <Block X="3" Y="2" Width="4" Height="1" />
		            <Block X="4" Y="1" Width="2" Height="1" />
		        </LevelPart>
		    </Parts>
		    <Objects>
		        <Object Type="Spawn" LevelPartIndex="0" X="1" Y="8" />
		        <Object Type="Spawn" LevelPartIndex="0" X="8" Y="8" />
		        <Object Type="Key" LevelPartIndex="0" X="0" Y="0" />
		        <Object Type="Door" LevelPartIndex="0" X="9" Y="0" />
		    </Objects>
		</Level>
	\end{verbatim}
	
	\item[Forgatókönyv]
	\begin{verbatim}

		levels/01.xml
		21
		p1 right
		p1 jump
		p2 left
		p2 jump
		go 20
		p1 jump
		p2 jump
		go 10
		p1 left
		p1 jump
		p2 right
		go 20
		p1 jump
		go 10
		p2 left
		go 20
		p2 stop
		p1 right
		go 35
		p1 left
		go 10
	\end{verbatim}
	
	\item[Elvárt kimenet] Amennyiben az első számú játékos sikeresen átmászott a legfelső blokkon jobbról balra, majd a legalsó levegőben lévő blokk alá állt be, és eközben kizárólag a második játékoson haladt keresztül blokkokon pedig nem, a teszt sikeresnek számít.
	
\end{description}

\newpage

\subsubsection{Játékos irányítása \#2}
\begin{description}
	
	\item[Leírás] A két játékos ugyancsak egy 1x1 pályarészből álló pályán van amikben több össze-vissza elhelyezett téglalap és háromszög alakú blokk is található. A két játékos minden irányban ugrál és mozog.
	
	\item[Ellenőrzött funkcionalitás] Ezzel a teszttel megfigyelhető hogy a játékosok egyszerű mozgatása és az ütközésvizsgálat a háromszög alakú blokkokkal a várt módon történnek.
	
	\item[Várható hibahelyek] Ennél a tesztnél elsősorban a játékosok és a háromszög alakú blokkok ütközése a kritikus pont. A játékosok akármelyik irányból is ütköznek a blokkokkal, semmi képp sem haladhatnak át rajta vagy "lóghatnak" bele. Amennyiben a háromszög alakú blokk, avagy a rámpa dőlésszöge meghaladja a 45 fokot, a játékosnak nem szabad hogy sikerüljön felsétálnia rajta.

	\item[Pálya]
	\begin{verbatim}
	
		------------
		|K D       |
		|      \   |
		|      \\  |
		|     /\\\ |
		|    //\\\\|
		|   XXXXXXX|
		|          |
		|\        /|
		|\\\  12 //|
		|XXXXXXXXXX|
		------------
	\end{verbatim}

	\item[Pálya leíró]
	\begin{verbatim}
	
		<?xml version="1.0" encoding="utf-8"?>
		<Level Width="1" Height="1">
		    <Parts>
		        <LevelPart X="0" Y="0">
		            <Block X="0" Y="9" Width="10" Height="1" />
		            <Block X="0" Y="7" Width="4" Height="2" Type="RightRamp" />
		            <Block X="8" Y="6" Width="2" Height="3" Type="LeftRamp" />
					
		            <Block X="3" Y="5" Width="7" Height="1" />
		            <Block X="3" Y="3" Width="3" Height="2" Type="LeftRamp" />
		            <Block X="6" Y="1" Width="4" Height="4" Type="RightRamp" />
					
		        </LevelPart>
		    </Parts>
		    <Objects>
		        <Object Type="Spawn" LevelPartIndex="0" X="5" Y="8" />
		        <Object Type="Spawn" LevelPartIndex="0" X="6" Y="8" />
		        <Object Type="Key" LevelPartIndex="0" X="0" Y="0" />
		        <Object Type="Door" LevelPartIndex="0" X="2" Y="0" />
		    </Objects>
		</Level>
	\end{verbatim}
	
	\item[Forgatókönyv]
	\begin{verbatim}

		levels/02.xml
		21
		p1 left
		p1 jump
		p2 right
		p2 jump
		go 20
		p1 right
		p1 jump
		p2 jump
		go 20
		p2 jump
		go 5
		p1 jump
		go 15
		p2 jump
		go 15
		p1 left
		go 5
		p2 jump
		go 30
		p1 right
		go 15
	\end{verbatim}
	
	\item[Elvárt kimenet] Amennyiben az első számú játékos sikeresen átmászott a legfelső háromszögön balról jobbra, és vissza tudott sétálni a kezdő ponthoz, miközben a második játékosnak többszöri próbálkozás után se sikerült felkapaszkodnia a 45 fokosnál meredekebb lejtőre, a teszt sikeres.
	
\end{description}

\newpage

\subsubsection{Kulcsok és ajtó}
\begin{description}
	
	\item[Leírás] A két játékos ismét egy  1x1 pályarészből álló pályán van amikben 3 kulcs, illetve az ajtó található. A két játékos minden irányban mozog.
	
	\item[Ellenőrzött funkcionalitás] Ezzel a teszttel megfigyelhető hogy a játékosok ajtóval való ütközése csak akkor vezet a pálya teljesítéséhez, ha nincs több kulcs a pályán, illetve a kulcsokkal való ütközéskor a kulcsok eltűnnek.
	
	\item[Várható hibahelyek] Ennél a tesztesetnél elsősorban a játékosok kulcsokkal illetve ajtóval való ütközése a kritikus pont. A játékosok akárhogyan is ütköznek az ajtóval, amíg nincs minden kulcs felvéve (tetszőleges játékos által) addig a játék folytatódik. Fontos, hogy mindkét játékos fel tudja venni a kulcsokat.

	\item[Pálya]
	\begin{verbatim}
	
		------------
		|          |
		|          |
		|          |
		|          |
		|          |
		|          |
		|          |
		|2  DK KK 1|
		|XXXXXXXXXX|
		|XXXXXXXXXX|
		------------	
	\end{verbatim}

	\item[Pálya leíró]
	\begin{verbatim}
	
		<?xml version="1.0" encoding="utf-8"?>
		<Level Width="2" Height="1">
		    <Parts>
		        <LevelPart X="0" Y="0">
		            <Block X="0" Y="8" Width="10" Height="2" />
		        </LevelPart>
		    </Parts>
		    <Objects>
		        <Object Type="Spawn" LevelPartIndex="0" X="9" Y="7" />
		        <Object Type="Spawn" LevelPartIndex="0" X="0" Y="7" />
		        <Object Type="Key" LevelPartIndex="0" X="7" Y="7" />
		        <Object Type="Key" LevelPartIndex="0" X="4" Y="7" />
		        <Object Type="Key" LevelPartIndex="0" X="6" Y="7" />
		        <Object Type="Door" LevelPartIndex="0" X="3" Y="7" />
		    </Objects>
		</Level>
	\end{verbatim}
	\newpage
	
	\item[Forgatókönyv]
	\begin{verbatim}

		levels/level03.xml
		20
		p2 right
		go 15
		p2 left
		go 10
		p1 left
		go 5
		p1 right
		go 5
		p2 right
		go 15
		p2 left
		go 10
		p1 left
		go 15
		p1 right
		go 10
		p2 right
		go 15
	\end{verbatim}
	
	\item[Elvárt kimenet] A második játékos átmegy az ajtón, felvéve az első kulcsot, majd az ajtó előtt oda-vissza lépked, míg az első játékos fel nem veszi az összes kulcsot (minden kulcs felvétele előtt legalább egyszer áthalad a játékos az ajtón). A teszt sikeresen zárul, ha mindaddig nem teljesített a pálya, míg minden kulcs felvételre nem kerül.
	
	\end{description}
\newpage

\subsubsection{Pályarész határok}
\begin{description}
	
	\item[Leírás] Egy egyszerű 2x2-1 pályarészből álló pályán való egyszerű mozgást tesztelünk. Miközben a pályarészeket néha-néha elcsúsztatjuk a játékosokkal legalább egyszer áthaladunk valamely pályarész határán majd később valamikor megakadunk egy másikon. A teszt során az egyik játékos egyszer "szakadékba" zuhan és visszakerül a kezdő pozíciójára, a másik pedig úgy sétál bele a lyukba hogy van alatta illeszkedő pályarész, és sikeresen átkerül rá.
	
	\item[Ellenőrzött funkcionalitás] Pályaelemek tili-toli mozgatása és keresztülhaladás pályarészeken.
	
	\item[Várható hibahelyek] Amennyiben valamely játékos nem hal meg amikor szakadékba zuhan, nem sikerül átmennie egy illeszkedő pályarészre, vagy épp sikerül átmennie egy nem illeszkedőre, a teszt sikertelen.

	\item[Pálya]
	\begin{verbatim}
	
		------------------------
		| K D      ||          |
		|          ||          |
		|          ||          |
		|          ||          |
		|          ||          |
		|          ||          |
		|          ||          |
		|   21     ||          |
		|XXXXXXXXXX||XXXXXXX  X|
		|XXXXXXXXXX||XXXXXXX  X|
		------------------------
		            ------------
		            |XXXXXXX  X|
		            |          |
		            |       /XX|
		            |      //  |
		            |     ///  |
		            |    ////  |
		            |   /////  |
		            |  //////  |
		            |XXXXXXXXXX|
		            |XXXXXXXXXX|
		            ------------
	\end{verbatim}

	\item[Pálya leíró]
	\begin{verbatim}
	
		<?xml version="1.0" encoding="utf-8"?>
		<Level Width="2" Height="2">
		    <Parts>
		        <LevelPart X="0" Y="0">
		            <Block X="0" Y="8" Width="10" Height="2" />
		        </LevelPart>
		        <LevelPart X="1" Y="0">
		            <Block X="0" Y="8" Width="7" Height="2" />
		            <Block X="9" Y="8" Width="1" Height="2" />
		        </LevelPart>
		        <LevelPart X="1" Y="1">
		            <Block X="0" Y="8" Width="10" Height="2" />
		            <Block X="1" Y="2" Width="7" Height="7" Type="LeftRamp" />
		            <Block X="8" Y="2" Width="2" Height="1" />
		            
		            <Block X="0" Y="0" Width="7" Height="1" />
		            <Block X="9" Y="0" Width="1" Height="1" />
		        </LevelPart>
		    </Parts>
		    <Objects>
		        <Object Type="Spawn" LevelPartIndex="0" X="4" Y="8" />
		        <Object Type="Spawn" LevelPartIndex="0" X="3" Y="8" />
		        <Object Type="Key" LevelPartIndex="0" X="1" Y="0" />
		        <Object Type="Door" LevelPartIndex="0" X="3" Y="0" />
		    </Objects>
		</Level>
	\end{verbatim}
	
	\item[Forgatókönyv]
	\begin{verbatim}

		levels/04.xml
		21
		p1 right
		p2 left
		go 20
		p2 right
		go 50
		slide down
		p1 stop
		p1 jump
		p2 left
		go 20
		slide left
		p2 right
		p1 jump
		go 20
		p1 left
		go 60
	\end{verbatim}
	
	\item[Elvárt kimenet] A játékosok csak az illeszkedő pályarészekre tudtak áthaladni, az első játékos sikeresen áthaladt egyik pályarészről a másikra felülről lefele és vissza is, a második ugyan ezt megpróbálva illeszkedő elem híján meghal. Az első játékos akkor is megpróbál áthaladni alulról felfele amikor már nincs illeszkedő pályaelem, de ekkor csak visszapattan mintha ott plafon lenne.
	
\end{description}

\newpage

\subsubsection{Főpróba}
\begin{description}
	
	\item[Leírás] Egy konkrét pálya végigvitelének tesztelése, a cél a program általános működésének vizsgálata.
	
	\item[Ellenőrzött funkcionalitás] Játékosok és pályaelemek mozgatása, halál, kulcsok felvétele, illetve kooperatív módú győzelem.
	
	\item[Várható hibahelyek] Az itt leírt funkcionalitásoknak már az előző tesztekben részleteztük a várható hibahelyeit.

	\item[Pálya]
	\begin{verbatim}
	
		------------
		|          |
		|          |
		|          |
		|          |
		|          |
		|    K     |
		|   XX     |
		|21 XXXX   |
		|XXXXXXXXXX|
		|XXXXXXXXXX|
		------------
		------------------------
		|XXX  XXXXX||          |
		|          ||          |
		|          ||          |
		|          ||          |
		|          ||          |
		|          ||          |
		|          ||          |
		|      D  K|| K        |
		|XXXXXXXXXX||XXX  XXXXX|
		|XXXXXXXXXX||XXX  XXXXX|
		------------------------
	\end{verbatim}

	\item[Pálya leíró]
	\begin{verbatim}
	
		<?xml version="1.0" encoding="utf-8"?>
		<Level Width="2" Height="2">
		    <Parts>
		        <LevelPart X="0" Y="0">
		            <Block X="0" Y="8" Width="10" Height="2" />
		            <Block X="3" Y="6" Width="2" Height="1" />
		            <Block X="3" Y="7" Width="4" Height="1" />
		        </LevelPart>
		        <LevelPart X="0" Y="1">
		            <Block X="0" Y="8" Width="10" Height="2" />
		            <Block X="0" Y="0" Width="3" Height="1" />
		            <Block X="5" Y="0" Width="5" Height="1" />
		        </LevelPart>
		        <LevelPart X="1" Y="1">
		            <Block X="0" Y="8" Width="3" Height="2" />
		            <Block X="5" Y="8" Width="5" Height="2" />
		        </LevelPart>
		    </Parts>
		    <Objects>
		        <Object Type="Spawn" LevelPartIndex="0" X="1" Y="7" />
		        <Object Type="Spawn" LevelPartIndex="0" X="0" Y="7" />
		        <Object Type="Key" LevelPartIndex="0" X="4" Y="5" />
		        <Object Type="Key" LevelPartIndex="1" X="9" Y="7" />
		        <Object Type="Key" LevelPartIndex="2" X="1" Y="7" />
		        <Object Type="Door" LevelPartIndex="1" X="6" Y="7" />
		    </Objects>
		</Level>
	\end{verbatim}
	
	\item[Forgatókönyv]
	\begin{verbatim}

		levels/05.xml
		21
		p1 right
		p1 jump
		go 20
		p2 right
		p2 jump
		go 20
		slide up
		go 25
		p1 jump
		go 13
		slide right
		go 52
	\end{verbatim}
	
	\item[Elvárt kimenet] A két játékos mozog, egy pályaelemet mozgatunk, a játékosok átmennek az előbb mozgatott pályaelembe, az egyikük (1-es) leesik egy olyan szakadékba, amelynek nincs másik pályaelemben folytatása és meghal. Ekkor a kezdőhelyén éled újra, és ismét elindul az előző útján, eközben mozgatunk egy pályaelemet és a 2-es számú játékos leesik az immár a helyén lévő végső pályaelembe. Itt felveszi az utolsó kulcsot, majd végül az 1-es játékos az ajtóhoz megy és vége a pályának. Az összes előző tesztekben lefektetett elvárásunk továbbra is él.
	
\end{description}

\newpage

\subsection{Tesztek jegyzőkönyvei}

\subsubsection{Játékos irányítása 1 - JÓ}

\begin{usecase}
	\usecaseentry{Tesztelő neve}{Biró}
	\usecaseentry{Teszt időpontja}{2012.04.12. 17:20}
\end{usecase}


\subsubsection{Játékos irányítása 2 - ROSSZ}

\begin{usecase}
	\usecaseentry{Tesztelő neve}{Biró}
	\usecaseentry{Teszt időpontja}{2012.04.12. 17:40}
	\usecaseentry{Teszt eredménye}{Hibajelenség: A játékos plafonnal való ütközéskor fent ragad egy rövid időre.}
	\usecaseentry{Lehetséges hibaok}{A játékos vertikális sebessége nem nullázódik a pálya tetejével való ütközéskör.}
	\usecaseentry{Változtatások}{A program megfelelő része javítva.}
\end{usecase}


\subsubsection{Kulcsok és ajtó - JÓ}

\begin{usecase}
	\usecaseentry{Tesztelő neve}{Magyar}
	\usecaseentry{Teszt időpontja}{2012.04.13. 20:10}
\end{usecase}

\subsubsection{Pályarész határok - JÓ}

\begin{usecase}
	\usecaseentry{Tesztelő neve}{Biró}
	\usecaseentry{Teszt időpontja}{2012.04.13. 22:15}
\end{usecase}

\subsubsection{Főpróba - JÓ}

\begin{usecase}
	\usecaseentry{Tesztelő neve}{Biró}
	\usecaseentry{Teszt időpontja}{2012.04.13. 22:25}
\end{usecase}


\subsection{Értékelés}
\begin{tabular}{|p{130pt}|c|p{150pt}|}
\hline 
\textbf{Tag neve} & \textbf{Munka százalékban} & \textbf{Aláírás}\\ 
\hline 
Biró Loránd & 22\% & \\ 
\hline 
Kanyó Tibor & 19.5\% & \\
\hline 
Magyar Dániel & 18.5\% & \\
\hline 
Tarjáni Tamás & 19.5\% & \\
\hline 
Vajsz Kornél & 20.5\% & \\
\hline 
\end{tabular} 
\newpage

\subsection{Napló}

\begin{journal}
\journalentry{2012.04.04. 13:30}{1.5}{Biró, Kanyó, Magyar, Tarjáni, Vajsz}{Értekezlet, lefektettük a prototípus egyes részeit ki fogja megvalósítani és milyen formában.}
\journalentry{2012.04.05. 15:30}{1}{Kanyó}{Dokumentum sablon elkészítése. Előzetes fájllista létrehozása a szkeleton alapján. Fordítási és futtatási leírás kiegészítése. Tesztek jegyzőkönyveihez sablon készítése.}

\journalentry{2012.04.06. 18:00}{3.0}{Magyar}{A \emph{model} package megírása}
\journalentry{2012.04.07. 12:00}{5.0}{Vajsz}{Megírtam a \emph{GameObject}-eket.}
\journalentry{2012.04.07. 13:30}{4.0}{Tarjáni}{A \emph{LevelPart} és \emph{Block} megvalósítása.}
\journalentry{2012.04.08. 11:00}{4.0}{Magyar}{A \emph{Vector2}, \emph{TriangleBounds} és a \emph{RectangleBounds} implementálása}
\journalentry{2012.04.08. 12:30}{1.0}{Biró}{A \emph{main} package osztályainak kezdetleges verziója elkészült.}
\journalentry{2012.04.09. 13:00}{3.5}{Kanyó}{A \emph{Level} osztály megírása.}
\journalentry{2012.04.11. 16:00}{2.5}{Biró}{A \emph{main} package véglegesítése és a játékos mozgásának tesztelése 1-2.}
\journalentry{2012.04.12. 17:30}{3.0}{Vajsz}{Újraírtam a \emph{Player} osztály ütközésvizsgálatát.}

\journalentry{2012.04.13. 20:00}{0.5}{Magyar}{Tesztelés.}
\journalentry{2012.04.13. 22:00}{0.5}{Biró}{Fennmaradó tesztek elvégzése.}
\journalentry{2012.04.12. 21:00}{1}{Kanyó}{A fájllista javítása.}
\journalentry{2012.04.12. 22:30}{0.5}{Tarjáni}{A dokumentum ellenőrzése, helyenként javítása.}
\journalentry{2012.04.15. 23:30}{1,5}{Magyar}{Tesztek jegyzőkönyveinek formalizálása. Dokumentáció ellenőrzése}

\end{journal}

\end{document}