%%%%%%%%%%%%%%%%%%%%%%%%%%%%%%%%%%%%%%%%%%%%%%%%%%%%%%%%%%%%%%%%%%%%%%%%%%%%%%%%
% Inicializálás                                                                %
%%%%%%%%%%%%%%%%%%%%%%%%%%%%%%%%%%%%%%%%%%%%%%%%%%%%%%%%%%%%%%%%%%%%%%%%%%%%%%%%

%%%%%%%%%%%%%%%%%%%%%%%%%%%%%%%%%%%%%%%%%%%%%%%%%%%%%%%%%%%%%%%%%%%%%%%%%%%%%%%%
% Papírméret, betűméret, margó, magyar karakterek                              %
%%%%%%%%%%%%%%%%%%%%%%%%%%%%%%%%%%%%%%%%%%%%%%%%%%%%%%%%%%%%%%%%%%%%%%%%%%%%%%%%

\documentclass[a4paper,12pt]{article}
\special{papersize=210mm,297mm}

\usepackage{anysize}
\marginsize{2.5cm}{2.5cm}{2.5cm}{2.5cm}

\usepackage[utf8]{inputenc}
\usepackage[magyar]{babel}

%%%%%%%%%%%%%%%%%%%%%%%%%%%%%%%%%%%%%%%%%%%%%%%%%%%%%%%%%%%%%%%%%%%%%%%%%%%%%%%%
% Fedlap inicializálása                                                        %
%%%%%%%%%%%%%%%%%%%%%%%%%%%%%%%%%%%%%%%%%%%%%%%%%%%%%%%%%%%%%%%%%%%%%%%%%%%%%%%%

\usepackage{fedlap}

\csapat{unexpected\_exceptions}{59}
\konzulens{Ferencz Endre}

\taga{Biró Loránd}{NCZAGL}{lol.kylerrr@gmail.com}
\tagb{Kanyó Tibor}{NXWUKE}{kanyo.tibi@gmail.com}
\tagc{Magyar Dániel}{SUFFGT}{samuraidanm@gmail.com}
\tagd{Tarjáni Tamás}{S499KV}{tarjanitomi@gmail.com}
\tage{Vajsz Kornél}{VUYNAW}{roncsipar@gmail.com}

%%%%%%%%%%%%%%%%%%%%%%%%%%%%%%%%%%%%%%%%%%%%%%%%%%%%%%%%%%%%%%%%%%%%%%%%%%%%%%%%
% Fejléc és lábléc                                                             %
%%%%%%%%%%%%%%%%%%%%%%%%%%%%%%%%%%%%%%%%%%%%%%%%%%%%%%%%%%%%%%%%%%%%%%%%%%%%%%%%

\usepackage{fancyhdr}

\setlength{\headheight}{1.4em}
\setlength{\headsep}{2em}

\fancyhf{}
\fancyhead[OL] { \leftmark{} }
\fancyhead[OR] { \tmpcsapat }
\fancyfoot[OC] { \thepage }
\fancyfoot[OR] { \tmpdatum }

\pagestyle{fancy}

%%%%%%%%%%%%%%%%%%%%%%%%%%%%%%%%%%%%%%%%%%%%%%%%%%%%%%%%%%%%%%%%%%%%%%%%%%%%%%%%
% Napló                                                                        %
%%%%%%%%%%%%%%%%%%%%%%%%%%%%%%%%%%%%%%%%%%%%%%%%%%%%%%%%%%%%%%%%%%%%%%%%%%%%%%%%

\usepackage{longtable}

\newenvironment{journal}
{
	\hbadness 10000
	\begin{longtable}{|p{60pt}|l|l|p{216pt}|}
	\hline
	\textbf{Kezdet} & \textbf{Időtartam} & \textbf{Résztvevők} & \textbf{Leírás} \\
	\hline
	\endfirsthead
	\hline
	\textbf{Kezdet} & \textbf{Időtartam} & \textbf{Résztvevők} & \textbf{Leírás} \\
	\hline
	\endhead
}
{
	\end{longtable}
}

\newcommand{\journalentry}[4]
{
	{#1} & {#2} óra & \parbox{50pt}{#3} & {#4} \\
	\hline
}

%%%%%%%%%%%%%%%%%%%%%%%%%%%%%%%%%%%%%%%%%%%%%%%%%%%%%%%%%%%%%%%%%%%%%%%%%%%%%%%%
% UseCase leírás                                                               %
%%%%%%%%%%%%%%%%%%%%%%%%%%%%%%%%%%%%%%%%%%%%%%%%%%%%%%%%%%%%%%%%%%%%%%%%%%%%%%%%

\newenvironment{usecase}
{
	\hbadness 10000
	\begin{longtable}[l]{|p{100pt}|p{328pt}|}
	\hline
	\endfirsthead
	\hline
	\endhead
}
{
	\hline
	\end{longtable}
}

\newcommand{\usecaseentry}[2]
{
	\hline
	\textbf{#1} & {#2}\\
}

%%%%%%%%%%%%%%%%%%%%%%%%%%%%%%%%%%%%%%%%%%%%%%%%%%%%%%%%%%%%%%%%%%%%%%%%%%%%%%%%
% FileList leírás                                                               %
%%%%%%%%%%%%%%%%%%%%%%%%%%%%%%%%%%%%%%%%%%%%%%%%%%%%%%%%%%%%%%%%%%%%%%%%%%%%%%%%

\newenvironment{filelist}
{
	\hbadness 10000

	\begin{longtable}{|p{145pt}|p{35pt}|p{63pt}|p{163pt}|}
	\hline
	\textbf{Fájl neve} & \textbf{Méret} & \textbf{Keletkezés ideje} & \textbf{Tartalom} \\
	\hline
	\endfirsthead
	\hline
	\textbf{Fájl neve} & \textbf{Méret} & \textbf{Keletkezés ideje} & \textbf{Tartalom} \\
	\hline
	\endhead
}
{
	\end{longtable}
}

\newcommand{\filelistentry}[4]
{
	{#1} & {#2} b & {#3} & {#4} \\
	\hline
}
%%%%%%%%%%%%%%%%%%%%%%%%%%%%%%%%%%%%%%%%%%%%%%%%%%%%%%%%%%%%%%%%%%%%%%%%%%%%%%%%
% Egyebek                                                                      %
%%%%%%%%%%%%%%%%%%%%%%%%%%%%%%%%%%%%%%%%%%%%%%%%%%%%%%%%%%%%%%%%%%%%%%%%%%%%%%%%

\usepackage{graphicx}		% Kepek beillesztesehez
\usepackage{epstopdf}		% EPS fajlok felismeresehez
\graphicspath{{Images/}}	% Az Images mappaban keresse a kepeket

\anyag{11. Grafikus felület specifikációja}
\datum{2012. április 22.}
\setcounter{section}{10}

%%%%%%%%%%%%%%%%%%%%%%%%%%%%%%%%%%%%%%%%%%%%%%%%%%%%%%%%%%%%%%%%%%%%%%%%%%%%%%%%
% Dokumentum                                                                   %
%%%%%%%%%%%%%%%%%%%%%%%%%%%%%%%%%%%%%%%%%%%%%%%%%%%%%%%%%%%%%%%%%%%%%%%%%%%%%%%%

\begin{document}

\fedlap

\section{Grafikus felület specifikációja}

\subsection{A grafikus interfész}

\subsection{A grafikus felület architektúrája}

\subsubsection{A felület működési elve}

\subsubsection{A felület osztály-struktúrája}

\subsection{A grafikus objektumok felsorolása}
Amelyik osztály leírásánál nem szerepel az Ősosztályok vagy az Interfészek pont, annak az osztálynak értelemszerűen nincs őse vagy nem valósít meg interfészt. Az összes, a leírásban szereplő attribútum és metódus publikus.

\subsubsection{Point}
	\begin{description}
		\item[Felelősség] \hfill \\
		Egy pontnak felel meg a koordináta-síkon.
		
		\item[Attribútumok]\hfill \\
		\textbf{\emph{float X}}: A pont x koordinátája.
		
		\textbf{\emph{float Y}}: A pont y koordinátája.		
				
		\item[Metódusok]\hfill \\
		-

	\end{description}
	
\subsubsection{Rectangle}
	\begin{description}
		\item[Felelősség] \hfill \\
		Egy téglalapot leíró osztály.
		
		\item[Ősosztályok] \hfill \\
		A {\itshape Point}	osztályból származik.		
		
		\item[Attribútumok]\hfill \\
		\textbf{\emph{float Height}}: A téglalap magassága.
		
		\textbf{\emph{float Width}}: A téglalap szélessége.		
				
		\item[Metódusok]\hfill \\
		-

	\end{description}
	
\subsubsection{Image}
	\begin{description}
		\item[Felelősség] \hfill \\
		Egy képet leíró osztály. Képes fájlból betölteni a képet, és bizonyos méretűre formázni.
		
		\item[Attribútumok]\hfill \\
		
		\textbf{\emph{BufferedImage BufferedImage}}: Egy java.awt.image.BufferedImage típusú változó, amiben a betöltött képet tároljuk.
		
		\textbf{\emph{int Height}}: A betöltött kép magassága.	
		
		\textbf{\emph{int Width}}: A betöltött kép szélessége.	
				
		\item[Metódusok]\hfill \\
		\textbf{\emph{LoadFromFile(String)}}: A paraméterként kapott elérési útról betölt egy képet.

	\end{description}
	
\subsubsection{RenderTransform}
	\begin{description}
		\item[Felelősség] \hfill \\
		Transzformációt leíró osztály.
		
		\item[Attribútumok]\hfill \\
		\textbf{\emph{float Scale}}: A transzformáció aránya, vagyis hogy hányszorosára növeljük a transzformálandó objektumot (képet, stb.).
		
		\textbf{\emph{float TranslateX}}: X koordináta szerinti transzformáció (az x tengely mentén mennyire nyomjuk össze vagy nyújtjuk a transzformált objektumot).	
		
		\textbf{\emph{float TranslateY}}: Y koordináta szerinti transzformáció.	
				
		\item[Metódusok]\hfill \\
		-

	\end{description}

\subsubsection{Renderer}
	\begin{description}
		\item[Felelősség] \hfill \\
		Lényegében ez az osztály felel a grafikus megjelenítésért, ez végzi a kirajzolásokat és a {\itshape RenderTransform} által leírt transzformációkat is.
		
		\item[Attribútumok]\hfill \\
		\textbf{\emph{int ScreenHeight}}: A játékot megjelenítő ablak magassága.
		
		\textbf{\emph{int ScreenWidth}}: A játékot megjelenítő ablak szélessége.		
				
		\item[Metódusok]\hfill \\
		\textbf{\emph{DrawImage(Rectangle, Image)}}: A paraméterként kapott képet kirajzolja a megadott téglalap által meghatározott helyre.
		
		\textbf{\emph{DrawImage(Rectangle, Rectangle, Image)}}: Az előző metódustól abban különbözik, hogy a második Rectangle paraméter segítségével kijelölhetjük a képből a megjelenítendő területet.
		
		\textbf{\emph{DrawPolygon(Point*[], Color)}}: Egy poligont rajzol fel, aminek a csúcsait egy, a pontokra mutató pointereket tartalmazó tömbben kapja meg. A poligon színét a {\itshape Color} típusú paraméter határozza meg.
		
		\textbf{\emph{DrawRectangle(Rectangle, Color)}}: Kirajzolja a paraméterként kapott téglalapot a megadott színnel.
		
		\textbf{\emph{DrawText(Point, String, int, Color)}}: Adott szöveget rajzol ki a kapott ponttól kezdődően az int paraméter által meghatározott betűmérettel és a megadott színnel.
		
		\textbf{\emph{SetTransform(RenderTransform)}}: Beállíthatjuk a használt kameratranszformációt.
						
	\end{description}
	


\subsection{Kapcsolat az alkalmazói rendszerrel}

\subsection{Napló}

\begin{journal}
\journalentry{2012.04.04. 13:30}{2}{Biró, Kanyó, Magyar, Tarjáni, Vajsz}{Értekezlet, döntés a grafikus megjelenítés mikéntjéről, a feladatok kiosztása.}
\journalentry{2012.04.22. 14:00}{1}{Tarjáni}{Dokumentumsablon, osztályleírások elkészítése.}


\end{journal}

\end{document}