%%%%%%%%%%%%%%%%%%%%%%%%%%%%%%%%%%%%%%%%%%%%%%%%%%%%%%%%%%%%%%%%%%%%%%%%%%%%%%%%
% Inicializálás                                                                %
%%%%%%%%%%%%%%%%%%%%%%%%%%%%%%%%%%%%%%%%%%%%%%%%%%%%%%%%%%%%%%%%%%%%%%%%%%%%%%%%

%%%%%%%%%%%%%%%%%%%%%%%%%%%%%%%%%%%%%%%%%%%%%%%%%%%%%%%%%%%%%%%%%%%%%%%%%%%%%%%%
% Papírméret, betűméret, margó, magyar karakterek                              %
%%%%%%%%%%%%%%%%%%%%%%%%%%%%%%%%%%%%%%%%%%%%%%%%%%%%%%%%%%%%%%%%%%%%%%%%%%%%%%%%

\documentclass[a4paper,12pt]{article}
\special{papersize=210mm,297mm}

\usepackage{anysize}
\marginsize{2.5cm}{2.5cm}{2.5cm}{2.5cm}

\usepackage[utf8]{inputenc}
\usepackage[magyar]{babel}

%%%%%%%%%%%%%%%%%%%%%%%%%%%%%%%%%%%%%%%%%%%%%%%%%%%%%%%%%%%%%%%%%%%%%%%%%%%%%%%%
% Fedlap inicializálása                                                        %
%%%%%%%%%%%%%%%%%%%%%%%%%%%%%%%%%%%%%%%%%%%%%%%%%%%%%%%%%%%%%%%%%%%%%%%%%%%%%%%%

\usepackage{fedlap}

\csapat{unexpected\_exceptions}{59}
\konzulens{Ferencz Endre}

\taga{Biró Loránd}{NCZAGL}{lol.kylerrr@gmail.com}
\tagb{Kanyó Tibor}{NXWUKE}{kanyo.tibi@gmail.com}
\tagc{Magyar Dániel}{SUFFGT}{samuraidanm@gmail.com}
\tagd{Tarjáni Tamás}{S499KV}{tarjanitomi@gmail.com}
\tage{Vajsz Kornél}{VUYNAW}{roncsipar@gmail.com}

%%%%%%%%%%%%%%%%%%%%%%%%%%%%%%%%%%%%%%%%%%%%%%%%%%%%%%%%%%%%%%%%%%%%%%%%%%%%%%%%
% Fejléc és lábléc                                                             %
%%%%%%%%%%%%%%%%%%%%%%%%%%%%%%%%%%%%%%%%%%%%%%%%%%%%%%%%%%%%%%%%%%%%%%%%%%%%%%%%

\usepackage{fancyhdr}

\setlength{\headheight}{1.4em}
\setlength{\headsep}{2em}

\fancyhf{}
\fancyhead[OL] { \leftmark{} }
\fancyhead[OR] { \tmpcsapat }
\fancyfoot[OC] { \thepage }
\fancyfoot[OR] { \tmpdatum }

\pagestyle{fancy}

%%%%%%%%%%%%%%%%%%%%%%%%%%%%%%%%%%%%%%%%%%%%%%%%%%%%%%%%%%%%%%%%%%%%%%%%%%%%%%%%
% Napló                                                                        %
%%%%%%%%%%%%%%%%%%%%%%%%%%%%%%%%%%%%%%%%%%%%%%%%%%%%%%%%%%%%%%%%%%%%%%%%%%%%%%%%

\usepackage{longtable}

\newenvironment{journal}
{
	\hbadness 10000
	\begin{longtable}{|p{60pt}|l|l|p{216pt}|}
	\hline
	\textbf{Kezdet} & \textbf{Időtartam} & \textbf{Résztvevők} & \textbf{Leírás} \\
	\hline
	\endfirsthead
	\hline
	\textbf{Kezdet} & \textbf{Időtartam} & \textbf{Résztvevők} & \textbf{Leírás} \\
	\hline
	\endhead
}
{
	\end{longtable}
}

\newcommand{\journalentry}[4]
{
	{#1} & {#2} óra & \parbox{50pt}{#3} & {#4} \\
	\hline
}

%%%%%%%%%%%%%%%%%%%%%%%%%%%%%%%%%%%%%%%%%%%%%%%%%%%%%%%%%%%%%%%%%%%%%%%%%%%%%%%%
% UseCase leírás                                                               %
%%%%%%%%%%%%%%%%%%%%%%%%%%%%%%%%%%%%%%%%%%%%%%%%%%%%%%%%%%%%%%%%%%%%%%%%%%%%%%%%

\newenvironment{usecase}
{
	\hbadness 10000
	\begin{longtable}[l]{|p{100pt}|p{328pt}|}
	\hline
	\endfirsthead
	\hline
	\endhead
}
{
	\hline
	\end{longtable}
}

\newcommand{\usecaseentry}[2]
{
	\hline
	\textbf{#1} & {#2}\\
}

%%%%%%%%%%%%%%%%%%%%%%%%%%%%%%%%%%%%%%%%%%%%%%%%%%%%%%%%%%%%%%%%%%%%%%%%%%%%%%%%
% FileList leírás                                                               %
%%%%%%%%%%%%%%%%%%%%%%%%%%%%%%%%%%%%%%%%%%%%%%%%%%%%%%%%%%%%%%%%%%%%%%%%%%%%%%%%

\newenvironment{filelist}
{
	\hbadness 10000

	\begin{longtable}{|p{145pt}|p{35pt}|p{63pt}|p{163pt}|}
	\hline
	\textbf{Fájl neve} & \textbf{Méret} & \textbf{Keletkezés ideje} & \textbf{Tartalom} \\
	\hline
	\endfirsthead
	\hline
	\textbf{Fájl neve} & \textbf{Méret} & \textbf{Keletkezés ideje} & \textbf{Tartalom} \\
	\hline
	\endhead
}
{
	\end{longtable}
}

\newcommand{\filelistentry}[4]
{
	{#1} & {#2} b & {#3} & {#4} \\
	\hline
}
%%%%%%%%%%%%%%%%%%%%%%%%%%%%%%%%%%%%%%%%%%%%%%%%%%%%%%%%%%%%%%%%%%%%%%%%%%%%%%%%
% Egyebek                                                                      %
%%%%%%%%%%%%%%%%%%%%%%%%%%%%%%%%%%%%%%%%%%%%%%%%%%%%%%%%%%%%%%%%%%%%%%%%%%%%%%%%

\usepackage{graphicx}		% Kepek beillesztesehez
\usepackage{epstopdf}		% EPS fajlok felismeresehez
\graphicspath{{Images/}}	% Az Images mappaban keresse a kepeket

\anyag{13. Grafikus változat beadása}
\datum{2012. május 6.}
\setcounter{section}{12}

%%%%%%%%%%%%%%%%%%%%%%%%%%%%%%%%%%%%%%%%%%%%%%%%%%%%%%%%%%%%%%%%%%%%%%%%%%%%%%%%
% Dokumentum                                                                   %
%%%%%%%%%%%%%%%%%%%%%%%%%%%%%%%%%%%%%%%%%%%%%%%%%%%%%%%%%%%%%%%%%%%%%%%%%%%%%%%%

\begin{document}

\fedlap

\section{Grafikus változat beadása}

\subsection{Fordítási és futtatási útmutató}

\subsubsection{Fájllista}
Az osztályok package-ekbe, így a fájlok mappákba vannak szervezve. 

Az \textbf{levels} mappa tartalma:
\begin{filelist}
	\filelistentry{level01.xml}{1149}{2012.04.30. 9:19}{A játék 1. pályája}
	\filelistentry{level02.xml}{909}{2012.04.30. 9:19}{2. pálya}
	\filelistentry{level03.xml}{956}{2012.04.30. 9:19}{3. pálya}
	\filelistentry{level04.xml}{3128}{2012.04.30. 9:19}{4.pálya}
	\filelistentry{ramp.xml}{507}{2012.04.30. 9:19}{Ferde blokkokat is tartalmazó pálya}
\end{filelist}


Az \textbf{src/main} mappa tartalma:
\begin{filelist}
	\filelistentry{GameCanvas.java}{1679}{2012.04.30. 9:19}{A rajzfelületet reprezentáló osztály, erre végezzük a kirajzolásokat}
	\filelistentry{GameFrame.java}{468}{2012.04.30. 9:19}{A játékablakot megvalósító osztály}	
	\filelistentry{KeyboardState.java}{1402}{2012.04.30. 9:19}{A billentyűzet eseményei alapján az adott képkockára érvényes billentyűzet-állapotot beállító osztály}
	\filelistentry{Main.java}{230}{2012.04.30. 9:19}{A program futását kezelő osztály}
\end{filelist}


A \textbf{src/model} mappa tartalma:
\begin{filelist}
	\filelistentry{BlockDescriptor.java}{1177}{2012.04.30. 9:19}{A faldarab leíró osztálya}
	\filelistentry{BlockType.java}{81}{2012.04.30. 9:19}{A faldarab típusait tartalmazó osztály}
	\filelistentry{LevelDescriptor.java}{2450}{2012.04.30. 9:19}{Egy pálya leíró osztálya}
	\filelistentry{LevelObjectDescriptor.java}{1050}{2012.04.30. 9:19}{Egy pályán elhelyezhető objektum leírója}
	\filelistentry{LevelObjectType.java}{76}{2012.04.30. 9:19}{A pályán elhelyezhető objektumok típusait tartalmazó osztály}
	\filelistentry{LevelPartDescriptor.java}{1005}{2012.04.30. 9:19}{Egy pályaelem paramétereit tartalmazó osztály}
\end{filelist}

A \textbf{src/gameLogic} mappa tartalma:
\begin{filelist}
	\filelistentry{Block.java}{786}{2012.04.30. 9:19}{A pálya építőelemét megvalósító osztály}
	\filelistentry{Direction.java}{81}{2012.04.30. 9:19}{Az irányokat tartalmazó osztály}
	\filelistentry{DirectionHelper.java}{437}{2012.04.30. 9:19}{A megadott iránynak az ellentettjét visszaadó osztály}
	\filelistentry{Door.java}{448}{2012.04.30. 9:19}{Az ajtót megvalósító osztály}
	\filelistentry{GameObject.java}{862}{2012.04.30. 9:19}{A játékban szereplő objektumok ősosztálya, és a legtöbb tulajdonságukat kezelő osztály}
	\filelistentry{IBounds.java}{1563}{2012.04.30. 9:19}{A blokkok ősosztálya}
	\filelistentry{KeyHolder.java}{644}{2012.04.30. 9:19}{A kulcsot megvalósító osztály}
	\filelistentry{Level.java}{5625}{2012.04.30. 9:19}{A pálya irányítását végző osztály}
	\filelistentry{LevelPart.java}{3966}{2012.04.30. 9:19}{Az egyes pályaelemeket vezérlő osztály}
	\filelistentry{LevelState.java}{301}{2012.04.30. 9:19}{A pálya állapotait tartalmazó felsorolás}
	\filelistentry{Player.java}{4906}{2012.04.30. 9:19}{A játékost kezelő osztály}
	\filelistentry{PlayerJumpState.java}{310}{2012.04.30. 9:19}{A játékos ugrásának állapotait tartalmazó osztály}
	\filelistentry{RectangleBounds.java}{4764}{2012.04.30. 9:19}{A négyszög alakú építőblokkok osztálya}
	\filelistentry{TriangleBounds.java}{3853}{2012.04.30. 9:19}{A háromszög alakú blokkok osztálya}
	\filelistentry{TriangleType.java}{82}{2012.04.30. 9:19}{A ferde talaj dőlésének irányát lehet megadni: balra vagy jobbra lejtsen}
	\filelistentry{Vector2.java}{704}{2012.04.30. 9:19}{2D vektorosztály}
\end{filelist}

Az \textbf{src/game} mappa tartalma:
\begin{filelist}
	\filelistentry{Animator.java}{1492}{2012.04.30. 9:19}{A játékosok animált mozgását megvalósító osztály}
	\filelistentry{AnimatorMode.java}{69}{2012.04.30. 9:19}{Az {\itshape Animator} osztály lehetséges módjait felsoroló osztály}	
	\filelistentry{Constants.java}{1015}{2012.04.30. 9:19}{A megjelenítéshez szükséges konstansokat tárolja}
	\filelistentry{ContinuityGame.java}{2453}{2012.04.30. 9:19}{A különböző jelenetek (scene-ek) közötti váltásokat kezeli}
	\filelistentry{ContinuityGameState.java}{71}{2012.04.30. 9:19}{Egy felsorolást tartalmazó osztály, melynek elemei azt adják meg, hogy a játékban vagyunk, vagy pedig menüben}
	\filelistentry{GameScene.java}{6213}{2012.04.30. 9:19}{Egy-egy pálya működéséért, megjelenítéséért felelős osztály}
	\filelistentry{GameSceneAction.java}{67}{2012.04.30. 9:19}{A pályán keletkező lehetséges események felsorolása}
	\filelistentry{GameSceneState.java}{72}{2012.04.30. 9:19}{A pálya lehetséges állapotait felsoroló osztály}
	\filelistentry{GameTime.java}{364}{2012.04.30. 9:19}{A mozgások szimulálásánál szükséges eltelt idők meghatározására szolgáló osztály}
	\filelistentry{IDrawableGameComponent.java}{171}{2012.04.30. 9:19}{A grafikus felületen megjeleníthető objektumok számára egy interfész.}
	\filelistentry{IGameComponent.java}{89}{2012.04.30. 9:19}{Interfész az időben változó objektumok számára}
	\filelistentry{Levels.java}{942}{2012.04.30. 9:19}{Az aktuális pálya sorszámát visszaadó és a mentést végző osztály}
	\filelistentry{MenuScene.java}{6326}{2012.04.30. 9:19}{A menü működéséért, megjelenítéséért felelős objektum}
	\filelistentry{MenuSceneAction.java}{83}{2012.04.30. 9:19}{A menüben kiadható eseményeket felsoroló osztály}
	\filelistentry{MenuSceneState.java}{102}{2012.04.30. 9:19}{A menü lehetséges állapotainak felsorolása}
\end{filelist}

Az \textbf{src/game/level} mappa tartalma:
\begin{filelist}
	\filelistentry{DoorComponent.java}{1252}{2012.04.30. 9:19}{Az ajtót megjelenítő osztály}
	\filelistentry{FrameRateCounterComponent.java}{946}{2012.04.30. 9:19}{Az aktuális FPS kiszámolásáért felelős}	
	\filelistentry{KeyComponent.java}{1286}{2012.04.30. 9:19}{A kulcs megjelenítéséért felelős osztály}
	\filelistentry{LevelPartComponent.java}{4471}{2012.04.30. 9:19}{Egy adott pályaelem megjelenítésért felelős osztály}
	\filelistentry{LevelScene.java}{4011}{2012.04.30. 9:19}{Egy adott pálya kezeléséért felelős osztály}
	\filelistentry{LevelSceneCamera.java}{4048}{2012.04.30. 9:19}{A teljes játéktérből a megfelelő részt láttatja}
	\filelistentry{LevelSceneCameraMode.java}{108}{2012.04.30. 9:19}{A lehetséges kameramódokat leíró osztály}
	\filelistentry{PlayerComponent.java}{4461}{2012.04.30. 9:19}{A játékos megjelenítését vezérlő osztály}
	\filelistentry{SpriteAnimation.java}{422}{2012.04.30. 9:19}{A pályaelemek mozgatását végző osztály}
	\filelistentry{SrpritePlayer.java}{1636}{2012.04.30. 9:19}{Egy játékos megjelenítéséért felelős}
\end{filelist}

Az \textbf{src/game/renderer} mappa tartalma:
\begin{filelist}
	\filelistentry{Image.java}{751}{2012.04.30. 9:19}{Egy képet leíró osztály}
	\filelistentry{Point.java}{438}{2012.04.30. 9:19}{Egy pontot megvalósító osztály}	
	\filelistentry{Rectangle.java}{367}{2012.04.30. 9:19}{egy téglalapot leíró osztály}
	\filelistentry{Renderer.java}{3549}{2012.04.30. 9:19}{A grafikus megjelenítésért felelős osztály}
	\filelistentry{RenderTransform.java}{810}{2012.04.30. 9:19}{Egy transzformációt leíró osztály}
\end{filelist}

A \textbf{src} mappa tartalma:
\begin{filelist}
	\filelistentry{manifest.txt}{61}{2012.04.30. 9:19}{A fordításhoz szükséges leíró fájl}
\end{filelist}


A gyökérben található fájlok:
\begin{filelist}
	\filelistentry{build.bat}{423}{2012.04.30. 9:19}{A fordítást indító fájl}
	\filelistentry{run.bat}{57}{2012.04.30. 9:19}{Az alkalmazást elindító fájl}
	\filelistentry{save.txt}{1}{2012.04.30. 9:19}{A befejezett pályák száma a játék folytatásához}
\end{filelist}

\newpage

\subsubsection{Fordítás és telepítés}
A fordítóprogram és annak működési módja majdnem teljesen megegyezik a 6.1.2 Fordítás nevű fejezetben leírtakkal. Pusztán az újabb package-eket kell hozzáadni a feldolgozandó listához. A \emph{build.bat} fájlt futtatva a konzolból, előáll a a lefordított jar fájl. A fájl az alábbi utasításokat tartalmazza:

\begin{verbatim}
@echo off
if not exist bin (mkdir bin)

echo Compiling classes...
javac src/main/*.java src/model/*.java src/gameLogic/*.java src/game/*.java src/game/renderer/*.java src/game/level/*.java -d bin -encoding UTF-8

cd bin
echo Creating jar...
jar cfm ../Continuity.jar ../src/manifest.txt main/*.class model/*.class game/*.class gameLogic/*.class game/renderer/*.class game/level/*.class
cd ..

echo Done
@echo on
\end{verbatim}

\subsubsection{Futtatás}
A futtatás megkönnyítése érdekében létrehoztunk egy batch fájlt: \emph{run.bat}. Ezen fájlt kell futtatni a főkönyvtárból. Ezen fájl az alábbi parancsot végrehajtva elindítja az elkészült jar fájlt: 

\begin{verbatim}
@echo off
java -jar ContinuityPrototype.jar
@echo on
\end{verbatim}

A játék ezt követően egy új ablakot nyit, melyben fut a grafikus felület.

\subsection{Értékelés}
\begin{tabular}{|p{130pt}|c|p{150pt}|}
\hline 
\textbf{Tag neve} & \textbf{Munka százalékban} & \textbf{Aláírás}\\ 
\hline 
Biró Loránd & 22\% & \\ 
\hline 
Kanyó Tibor & 20.5\% & \\
\hline 
Magyar Dániel & 18\% & \\
\hline 
Tarjáni Tamás & 20\% & \\
\hline 
Vajsz Kornél & 19.5\% & \\
\hline 
\end{tabular} 
\newpage

\subsection{Napló}

\begin{journal}
\journalentry{2012.04.25. 13:30}{1}{Biró, Kanyó, Magyar, Tarjáni, Vajsz}{Értekezlet, a grafikus változathoz tartozó feladatok felosztása.}
\journalentry{2012.05.04. 14:40}{5}{Tarjáni}{\emph{Renderer} részhez tartozó osztályok implementálása}
\journalentry{2012.05.05. 12:30}{0.5}{Biró}{Alkalmazás készítése a dokumentumok automatikus összefűzésére}
\journalentry{2012.05.05. 13:30}{4}{Kanyó}{Néhány, a pályához tartozó komponens implementálása}
\journalentry{2012.05.05. 17:15}{3}{Tarjáni}{Ezen dokumentáció előzetes elkészítése}
\journalentry{2012.05.05. 18:30}{4.5}{Biró}{Elkészítettem a \emph{GameScene} és a \emph{MenuScene} osztályokat}
\journalentry{2012.05.05. 19:00}{2}{Tarjáni}{A félév végi összefoglaló dokumentum megírása}
\journalentry{2012.05.05. 19:30}{1.5}{Magyar}{Kód kommentezése}
\journalentry{2012.05.05. 19:40}{0.5}{Biró}{Alkalmazás készítése az egyénenkénti munkaórák automatikus összeszámolásához}
\journalentry{2012.05.06. 10:10}{3.5}{Biró}{Létrehoztam a \emph{ContinuityGame}, illetve az \emph{Animator}, valamint a hozzájuk tartozó osztályokat}
\journalentry{2012.05.06. 12:30}{3}{Tarjáni}{További osztályok implementálása}
\journalentry{2012.05.06. 15:30}{4.5}{Kanyó}{A hiányzó komponensek megírása}
\journalentry{2012.05.06. 16:00}{1.5}{Magyar}{A képi háttéranyagok egy részének begyűjtése illetve másik részének létrehozása}
\journalentry{2012.05.06. 16:30}{1}{Biró}{Kommentezés}

\journalentry{2012.05.06. 18:00}{7}{Magyar}{}
\journalentry{2012.05.06. 12:00}{1}{Vajsz}{Kód kommentezése}
\journalentry{2012.05.06. 18:30}{4}{Tarjáni}{}
\journalentry{2012.05.06. 12:30}{5}{Biró}{}
\journalentry{2012.05.06. 20:30}{0.5}{Kanyó}{Program készítése a forrássorok összeszámlálására}
\journalentry{2012.05.06. 23:30}{1}{Kanyó}{A dokumentumok nyelvtani ellenőrzése}


\end{journal}

\end{document}