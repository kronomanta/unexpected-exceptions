%%%%%%%%%%%%%%%%%%%%%%%%%%%%%%%%%%%%%%%%%%%%%%%%%%%%%%%%%%%%%%%%%%%%%%%%%%%%%%%%
% Inicializálás                                                                %
%%%%%%%%%%%%%%%%%%%%%%%%%%%%%%%%%%%%%%%%%%%%%%%%%%%%%%%%%%%%%%%%%%%%%%%%%%%%%%%%

%%%%%%%%%%%%%%%%%%%%%%%%%%%%%%%%%%%%%%%%%%%%%%%%%%%%%%%%%%%%%%%%%%%%%%%%%%%%%%%%
% Papírméret, betűméret, margó, magyar karakterek                              %
%%%%%%%%%%%%%%%%%%%%%%%%%%%%%%%%%%%%%%%%%%%%%%%%%%%%%%%%%%%%%%%%%%%%%%%%%%%%%%%%

\documentclass[a4paper,12pt]{article}
\special{papersize=210mm,297mm}

\usepackage{anysize}
\marginsize{2.5cm}{2.5cm}{2.5cm}{2.5cm}

\usepackage[utf8]{inputenc}
\usepackage[magyar]{babel}

%%%%%%%%%%%%%%%%%%%%%%%%%%%%%%%%%%%%%%%%%%%%%%%%%%%%%%%%%%%%%%%%%%%%%%%%%%%%%%%%
% Fedlap inicializálása                                                        %
%%%%%%%%%%%%%%%%%%%%%%%%%%%%%%%%%%%%%%%%%%%%%%%%%%%%%%%%%%%%%%%%%%%%%%%%%%%%%%%%

\usepackage{fedlap}

\csapat{unexpected\_exceptions}{59}
\konzulens{Ferencz Endre}

\taga{Biró Loránd}{NCZAGL}{lol.kylerrr@gmail.com}
\tagb{Kanyó Tibor}{NXWUKE}{kanyo.tibi@gmail.com}
\tagc{Magyar Dániel}{SUFFGT}{samuraidanm@gmail.com}
\tagd{Tarjáni Tamás}{S499KV}{tarjanitomi@gmail.com}
\tage{Vajsz Kornél}{VUYNAW}{roncsipar@gmail.com}

%%%%%%%%%%%%%%%%%%%%%%%%%%%%%%%%%%%%%%%%%%%%%%%%%%%%%%%%%%%%%%%%%%%%%%%%%%%%%%%%
% Fejléc és lábléc                                                             %
%%%%%%%%%%%%%%%%%%%%%%%%%%%%%%%%%%%%%%%%%%%%%%%%%%%%%%%%%%%%%%%%%%%%%%%%%%%%%%%%

\usepackage{fancyhdr}

\setlength{\headheight}{1.4em}
\setlength{\headsep}{2em}

\fancyhf{}
\fancyhead[OL] { \leftmark{} }
\fancyhead[OR] { \tmpcsapat }
\fancyfoot[OC] { \thepage }
\fancyfoot[OR] { \tmpdatum }

\pagestyle{fancy}

%%%%%%%%%%%%%%%%%%%%%%%%%%%%%%%%%%%%%%%%%%%%%%%%%%%%%%%%%%%%%%%%%%%%%%%%%%%%%%%%
% Napló                                                                        %
%%%%%%%%%%%%%%%%%%%%%%%%%%%%%%%%%%%%%%%%%%%%%%%%%%%%%%%%%%%%%%%%%%%%%%%%%%%%%%%%

\usepackage{longtable}

\newenvironment{journal}
{
	\hbadness 10000
	\begin{longtable}{|p{60pt}|l|l|p{216pt}|}
	\hline
	\textbf{Kezdet} & \textbf{Időtartam} & \textbf{Résztvevők} & \textbf{Leírás} \\
	\hline
	\endfirsthead
	\hline
	\textbf{Kezdet} & \textbf{Időtartam} & \textbf{Résztvevők} & \textbf{Leírás} \\
	\hline
	\endhead
}
{
	\end{longtable}
}

\newcommand{\journalentry}[4]
{
	{#1} & {#2} óra & \parbox{50pt}{#3} & {#4} \\
	\hline
}

%%%%%%%%%%%%%%%%%%%%%%%%%%%%%%%%%%%%%%%%%%%%%%%%%%%%%%%%%%%%%%%%%%%%%%%%%%%%%%%%
% UseCase leírás                                                               %
%%%%%%%%%%%%%%%%%%%%%%%%%%%%%%%%%%%%%%%%%%%%%%%%%%%%%%%%%%%%%%%%%%%%%%%%%%%%%%%%

\newenvironment{usecase}
{
	\hbadness 10000
	\begin{longtable}[l]{|p{100pt}|p{328pt}|}
	\hline
	\endfirsthead
	\hline
	\endhead
}
{
	\hline
	\end{longtable}
}

\newcommand{\usecaseentry}[2]
{
	\hline
	\textbf{#1} & {#2}\\
}

%%%%%%%%%%%%%%%%%%%%%%%%%%%%%%%%%%%%%%%%%%%%%%%%%%%%%%%%%%%%%%%%%%%%%%%%%%%%%%%%
% FileList leírás                                                               %
%%%%%%%%%%%%%%%%%%%%%%%%%%%%%%%%%%%%%%%%%%%%%%%%%%%%%%%%%%%%%%%%%%%%%%%%%%%%%%%%

\newenvironment{filelist}
{
	\hbadness 10000

	\begin{longtable}{|p{145pt}|p{35pt}|p{63pt}|p{163pt}|}
	\hline
	\textbf{Fájl neve} & \textbf{Méret} & \textbf{Keletkezés ideje} & \textbf{Tartalom} \\
	\hline
	\endfirsthead
	\hline
	\textbf{Fájl neve} & \textbf{Méret} & \textbf{Keletkezés ideje} & \textbf{Tartalom} \\
	\hline
	\endhead
}
{
	\end{longtable}
}

\newcommand{\filelistentry}[4]
{
	{#1} & {#2} b & {#3} & {#4} \\
	\hline
}
%%%%%%%%%%%%%%%%%%%%%%%%%%%%%%%%%%%%%%%%%%%%%%%%%%%%%%%%%%%%%%%%%%%%%%%%%%%%%%%%
% Egyebek                                                                      %
%%%%%%%%%%%%%%%%%%%%%%%%%%%%%%%%%%%%%%%%%%%%%%%%%%%%%%%%%%%%%%%%%%%%%%%%%%%%%%%%

\usepackage{graphicx}		% Kepek beillesztesehez
\usepackage{epstopdf}		% EPS fajlok felismeresehez
\graphicspath{{Images/}}	% Az Images mappaban keresse a kepeket

\anyag{6. Szkeleton beadás}
\datum{2012. március 19.}
\setcounter{section}{5}

%%%%%%%%%%%%%%%%%%%%%%%%%%%%%%%%%%%%%%%%%%%%%%%%%%%%%%%%%%%%%%%%%%%%%%%%%%%%%%%%
% Dokumentum                                                                   %
%%%%%%%%%%%%%%%%%%%%%%%%%%%%%%%%%%%%%%%%%%%%%%%%%%%%%%%%%%%%%%%%%%%%%%%%%%%%%%%%

\begin{document}

\fedlap

\section{Szkeleton beadás}

\subsection{Fordítási és futtatási útmutató}

\subsubsection{Fájllista}
Az osztályok package-ekbe, így a fájlok mappákba vannak szervezve.

Az \textbf{src/main} mappa tartalma:
\begin{filelist}
	\filelistentry{Main.java}{1778}{2012.03.17. 17:10}{A program futását kezelő osztály}
	\filelistentry{SkeletonHelper.java}{2881}{2012.03.17. 17:12}{Információk kiírását segítő osztály.}
\end{filelist}

A \textbf{src/model} mappa tartalma:
\begin{filelist}
	\filelistentry{BlockDescriptor.java}{836}{2012.03.17. 17:15}{Magát a faldarab leíró osztálya}
	\filelistentry{BlockType.java}{81}{2012.03.17. 17:15}{A faldarab típusait tartalmazó osztály}
	\filelistentry{LevelDescriptor.java}{960}{2012.03.17. 17:18}{Egy pálya leíró osztálya}
	\filelistentry{LevelObjectDescriptor.java}{817}{2012.03.17. 17:18}{Egy pályán elhelyezhető objektum leírója}
	\filelistentry{LevelObjectType.java}{76}{2012.03.17. 17:20}{A pályán elhelyezhető objektumok típusait tartalmazó osztály}
	\filelistentry{LevelPartDescriptor.java}{702}{2012.03.17. 17:20}{Egy pályaelem paramétereit tartalmazó osztály}
\end{filelist}

A \textbf{src/gameLogic} mappa tartalma:
\begin{filelist}
	\filelistentry{Block.java}{536}{2012.03.17. 17:24}{A pálya építőelemét megvalósító osztály}
	\filelistentry{Direction.java}{81}{2012.03.17. 17:24}{Az irányokat tartalmazó osztály}
	\filelistentry{Door.java}{321}{2012.03.17. 17:25}{Az ajtót megvalósító osztály}
	\filelistentry{GameObject.java}{1100}{2012.03.17. 17:20}{A játékban szereplő objektumok ősosztálya, és a legtöbb tulajdonságukat kezelő osztály}
	\filelistentry{IBounds.java}{1530}{2012.03.18. 22:09}{A blokkok ősosztálya}
	\filelistentry{KeyHolder.java}{528}{2012.03.17. 17:27}{A kulcsot megvalósító osztály}
	\filelistentry{Level.java}{2314}{2012.03.17. 17:26}{A pálya irányítását végző osztály}
	\filelistentry{LevelPart.java}{2499}{2012.03.17. 17:32}{Az egyes pályaelemeket vezérlő osztály}
	\filelistentry{Player.java}{3255}{2012.03.17. 17:20}{A játékos kezelő osztály}
	\filelistentry{PlayerJumpState.java}{310}{2012.03.17. 17:23}{A játékos ugrásának állapotait tartalmazó osztály}
	\filelistentry{RectangleBounds.java}{3231}{2012.03.18. 22:09}{A négyszög alakú építőblokkok osztálya}
	\filelistentry{TriangleBounds.java}{3619}{2012.03.18. 22:09}{A háromszög alakú blokkok osztálya}
	\filelistentry{Vector2.java}{768}{2012.03.17. 17:35}{2D vektorosztály}
\end{filelist}

A \textbf{src} mappa tartalma:
\begin{filelist}
	\filelistentry{manifest.txt}{61}{2012.03.17. 19:30}{A fordításhoz szükséges leíró fájl}
\end{filelist}

A gyökérben található fájlok:
\begin{filelist}
	\filelistentry{build.bat}{314}{2012.03.17. 19:30}{A fordítást indító fájl}
	\filelistentry{run.bat}{53}{2012.03.17. 19:30}{Az alkalmazást elindító fájl}
\end{filelist}

\newpage

\subsubsection{Fordítás}
A fordítást intéző parancsok a \emph{make.bat} batch fájlba vannak gyűjtve, így a futtatható \emph{jar} fájl előállításához mindössze ezt kell elindítani. Előfeltétele hogy a JDK-val feltelepült \emph{java.exe}, \emph{javac.exe} és \emph{jar.exe} állományok elérhetőek legyenek tetszőleges könyvtárból. A fájl az alábbi utasításokat tartalmazza:

\begin{verbatim}
@echo off
if not exist bin (mkdir bin)

echo Compiling classes...
javac src/main/*.java src/model/*.java src/gameLogic/*.java -d bin
        -encoding UTF-8

cd bin
echo Creating jar...
jar cfm ../ContinuitySkeleton.jar ../src/manifest.txt main/*.class
        model/*.class gameLogic/*.class
cd ..

echo Done
@echo on
\end{verbatim}

\subsubsection{Futtatás}

A futtatás megkönnyítése érdekében létrehoztunk egy batch fájlt: \emph{run.bat}. Ezen fájlt kell futtatni a főkönyvtárából.

Ezen fájl az alábbi parancsot végrehajtva elindítja az elkészült jar fájlt: 
\begin{verbatim}
@echo off
java -jar ContinuitySkeleton.jar
@echo on
\end{verbatim}

A szkeleton egy konzolos programként fog futni.

\subsection{Értékelés}
\begin{tabular}{|p{130pt}|c|p{150pt}|}
\hline 
\textbf{Tag neve} & \textbf{Munka százalékban} & \textbf{Aláírás}\\ 
\hline 
Biró Loránd & 22\% & \\ 
\hline 
Kanyó Tibor & 19\% & \\
\hline 
Magyar Dániel & 19\% & \\
\hline 
Tarjáni Tamás & 19\% & \\
\hline 
Vajsz Kornél & 21\% & \\
\hline 
\end{tabular} 

\newpage

\subsection{Napló}

\begin{journal}

\journalentry{2012.03.17. 17:00}{6}{Biró}{Szkeleton írása.}
\journalentry{2012.03.18. 14:00}{1}{Tarjáni}{Dokumentum sablon elkészítése, fájllista elkezdése.}
\journalentry{2012.03.18. 17:00}{3}{Kanyó}{Fájllista létrehozása, igazítása. A lista sablonjának javítása. Batch fájlok létrehozása a futtatáshoz, fordításhoz és JDK beállításához. A Fordítás és Futtatás rész megírása.}
\journalentry{2012.03.18. 20:00}{2}{Vajsz}{IBounds.java, RectangleBounds.java, TriangleBounds.java kidolgozása.}
\journalentry{2012.03.19. 08:00}{1}{Magyar}{Dokumentáció, kommentezés kiegészítése.}

\end{journal}

\end{document}